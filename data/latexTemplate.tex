\title{A Very Simple \LaTeXe{} Template}
\author{Matt Goldberg}
\date{\today}

\documentclass[12pt]{article}

\usepackage[margin=0.5in]{geometry}
\usepackage{enumerate}
\usepackage{amsmath, amssymb, amsthm, mathtools}

% Statistics
\newcommand{\E}[1]{\mathbb{E}\left[ #1 \right]}
\newcommand{\var}[1]{\mathrm{Var}\left[ #1 \right]}
\newcommand{\cov}[2]{\mathrm{Cov}\left[ #1 , #2 \right]}
\newcommand{\iid}{\textrm{ i.i.d.}}
\newcommand{\simiid}{\overset\iid\sim}
% Distributions
\newcommand{\Norm}[2]{\mathrm{\mathcal{N}\left(#1, #2\right)}}
\newcommand{\Bin}[2]{\mathrm{Bin\left(#1, #2\right)}}
\newcommand{\Bern}[1]{\mathrm{Bern\left( #1 \right)}}
\newcommand{\Pois}[1]{\mathrm{Pois\left( #1 \right)}}
\newcommand{\Expo}[1]{\mathrm{Expo\left( #1 \right)}}
\newcommand{\GammaDist}[2]{\mathrm{Gamma\left( #1, #2 \right)}}
% Sets
\newcommand{\R}{\mathbb{Z}} % reals
\newcommand{\Z}{\mathbb{Z}} % integers
\newcommand{\N}{\mathbb{N}} % natural numbers
\newcommand{\F}[1]{\mathcal{F}_{#1}}
% argmin, argmax
\DeclareMathOperator*{\argmin}{arg\,min}
\DeclareMathOperator*{\argmax}{arg\,max}
% if/otherwise for cases environment
\newcommand{\myif}{& \textrm{if }}
\newcommand{\myotherwise}{& \textrm{otherwise.}}


\begin{document}
\maketitle

\begin{abstract}
This is the paper's abstract \ldots
\end{abstract}

\section{Introduction}
This is time for all good men to come to the aid of their party!

\paragraph{Outline}
The remainder of this article is organized as follows.
Section~\ref{previous work} gives account of previous work.
Our new and exciting results are described in Section~\ref{results}.
Finally, Section~\ref{conclusions} gives the conclusions.

\section{Previous work}\label{previous work}
A much longer \LaTeXe{} example was written by Gil~\cite{Gil:02}.

\section{Results}\label{results}
In this section we describe the results.

\section{Conclusions}\label{conclusions}
We worked hard, and achieved very little.

\bibliographystyle{abbrv}
\bibliography{simple}

\end{document}
